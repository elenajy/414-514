\documentclass[12pt,oneside]{amsart}
\linespread{2.4}
\usepackage[utf8]{inputenc}

\title{Math 414/514 Homework 6}
\author{Elena Yang}
\date{March 24, 2021}

% Packages
\usepackage[T1]{fontenc}
\usepackage{amsmath,amsfonts,amssymb,amsthm}
\usepackage[letterpaper,margin=1.5in]{geometry}
\usepackage[pagebackref]{hyperref}
\usepackage{booktabs}
\usepackage{enumitem}
\usepackage{fancyhdr}
\usepackage{hyperref}
\usepackage{mathtools}
\usepackage{comment}
\pagestyle{fancy}

\newtheorem{theorem}[equation]{Theorem}
\newtheorem{claim}[equation]{Claim}
\newtheorem{lemma}[equation]{Theorem}
\newtheorem{corollary}[equation]{Theorem}
\newtheorem{conjecture}[equation]{Conjecture}
\newtheorem{question}[equation]{Question}
\theoremstyle{definition}
\newtheorem{definition}[equation]{Definition}
\theoremstyle{remark}
\newtheorem{exer}{Exercise}
\newtheorem{remark}[equation]{Remark}
\newtheorem{example}[equation]{Example}
\numberwithin{equation}{exer}
\newenvironment{answer}{\bigskip\noindent\emph{Answer.}}{\hfill$\diamond$\newline}

\begin{document}
\maketitle
\begin{exer}
a) Lebl 6.1.4: Suppose $\{f_n\}$ and $\{g_n\}$ defined on some set $A$ converge to $f$ and $g$ respectively pointwise. Show that $\{f_n+g_n\}$ converges pointwise $f+g$. \\
b) Lebl 6.1.5: Suppose $\{f_n\}$ and $\{g_n\}$ defined on some set $A$ converge to $f$ and $g$ respectively uniformly on $A$. Show that $\{f_n+g_n\}$ converges uniformly to $f+g$ on $A$.
\end{exer}
\begin{proof}
a) By definition 1.1, for every $x\in S$ we have
$$f=\lim_{n\to\infty}f_n(x)$$
$$g=\lim_{n\to\infty}g_n(x)$$
Analyzing $\{f_n+g_n\}$ on $A$ we have
$$(f+g)(x)=\lim_{n\to\infty}\{f_n(x)+g_n(x)\}$$
$$=\lim_{n\to\infty}f_n(x)+\lim_{n\to\infty}g_n(x)$$
$$=f+g $$
b) Let $\epsilon>0$ be given. Then by definition 1.2, there exists $N\in\mathbb{N}$ such that for all $n\geq N$
$$|f_n(x)-f(x)|<\frac{\epsilon}{2}$$
$$|g_n(x)-g(x)|<\frac{\epsilon}{2}$$
for all $x\in\mathbb{A}$. 
Then, for all $n\geq N$ and $x\in A$, we have
$$|(f_n+g_n)(x)-(f+g)(x)|=|(f_n(x)+g_n(x))-(f(x)+g(x))|$$
$$\leq|f_n(x)-f(x)|+|g_n(x)-g(x)|$$
$$<\frac{\epsilon}{2}+\frac{\epsilon}{2}=\epsilon$$
Thus by definition $\{f_n+g_n\}$ converges uniformly to $f+g$ on $A$.
\end{proof}
\begin{definition}
For every $n\in\mathbb{N}$ let $f_n:S\rightarrow\mathbb{R}$ be a function. We say the sequence $\{f_n\}_{n=1}^\infty$ converges pointwise to $f:S\rightarrow\mathbb{R}$, if for every $x\in S$ we have 
$$f(x)=\lim_{n\to\infty}f_n(x)$$
\end{definition}
\begin{definition}
Let $f_n:S\rightarrow\mathbb{R}$ and $f:S\rightarrow\mathbb{R}$ be functions. We say the sequence $\{f_n\}$ converges uniformly to $f$, if for every $\epsilon>0$ there exists an $N\in\mathbb{N}$ such that for all $n\geq N$ we have 
$$|f_n(x)-f(x)|<\epsilon$$ for all $x\in S$.
\end{definition}

\newpage

\begin{exer}
Lebl 6.1.6: Find an example of a sequence of functions $\{f_n\}$ and $\{g_n\}$ that converge uniformly to some $f$ and $g$ on some set $A$ but such that $\{f_ng_n\}$ (the multiple) does not converge uniformly to $fg$ on $A$. Hint: Let $A:=\mathbb{R}$, let $f(x):=g(x):=x$. You can even pick $f_n=g_n$.
\end{exer}
\begin{answer}
Let $A=\mathbb{R}$ and let $f_n=g_n=x$. Then let $f_n(x)=g_n(x)=x+\frac{1}{n}$. \\
First, we will show that $f_n(x)=g_n(x)=x+\frac{1}{n}$ converges uniformly. Let $\epsilon>0$. Let $N\in\mathbb{N}$ exist such that for every $n\geq N$ we have:
$$|f_n(x)-f(x)|=|x+\frac{1}{n}-x|=\frac{1}{n}\leq\frac{1}{N}<\epsilon$$
Now, we will show that $\{f_ng_n\}$ does not converge uniformly. Let $\epsilon>0$. Let $N\in\mathbb{N}$ exist such that for every $n\geq N$ we have:
$$|f_ng_n(x)-f(x)g(x)|=|(x^2+\frac{2}{n}x+\frac{1}{n^2})-x^2|=\frac{2}{n}|x|+\frac{1}{n^2}>\epsilon$$
for $|x|\geq\frac{n\epsilon}{2}$. Thus by definition 1.2, $\{f_ng_n\}$ does not converge uniformly.
\end{answer}

\newpage

\begin{exer}
Lebl 6.1.10: Let $\{f_n\}$ be a sequence of functions defined on $[0,1]$. Suppose there exists a sequence of distinct numbers $x_n\in[0,1]$ such that 
$$f_n(x_n)=1$$
Prove or disprove the following statements: \\
a) True or false: There exists $\{f_n\}$ as above that converges to 0 pointwise. \\
b) True or false: There exists $\{f_n\}$ as above that converges to 0 uniformly on $[0,1]$.
\end{exer}
\begin{answer}
a) True. Let
\begin{equation*}
f_n(x) = \left\{
        \begin{array}{ll}
            2nx & \quad x\in[0,\frac{1}{2n}] \\
            2-2nx & \quad x\in[\frac{1}{2n},\frac{1}{n}] \\
            0 & \quad x\in[\frac{1}{n},1]
        \end{array}
    \right.
\end{equation*}
This function converges to 0 pointwise but $f_n(\frac{1}{2n})=1$.
\end{answer}
b) False.
\begin{proof}
Using definition 3.1, we see that $\{f_n\}$ converges to $0$ uniformly if the following holds true:
$$\lim_{n\to\infty}||f_n-0||_u=0$$
Rewriting the expression we see that:
$$\lim_{n\to\infty}||f_n-0||_u=\lim_{n\to\infty}\sup(f_n-0)\geq\lim_{n\to\infty}1\neq0$$
Therefore there does not exist $\{f_n\}$ that converges to $0$ uniformly.
\end{proof}
\begin{definition}
Proposition 6.1.10 states: A sequence of bounded functions $f_n:S\rightarrow\mathbb{R}$ converges uniformly to $f:S\rightarrow\mathbb{R}$ if and only if 
$$\lim_{n\to\infty}||f_n-f||_u=0$$
\end{definition}

\newpage

%\begin{exer}
%Lebl 6.2.12: Find a sequence of Riemann integrable functions $f_n:[0,1]\rightarrow\mathbb{R}$ such that $\{f_n\}$ converges to zero pointwise, and such that \\
%a) $\{\int_{0}^{1}f_n\}_{n=1}^{\infty}$ increases without bound, \\
%b) $\{\int_{0}^{1}f_n\}_{n=1}^{\infty}$ is the sequence $-1,1,-1,1,-1,1,...$
%\end{exer}

\begin{exer}
Lebl 6.2.7: For a continuously differentiable function $f:[a,b]\rightarrow\mathbb{R}$, define $$||f||_{C^1}:=||f||_u+||f'||_u$$Suppose $\{f_n\}$ is a sequence of continuously differentiable functions such that for every $\epsilon>0$, there exists an $M$ such that for all $n,k\geq M$ we have $$||f_n-f_k||_{C^1}<\epsilon$$Show that $\{f_n\}$ converges uniformly to some continuously differentiable function $f:[a,b]\rightarrow\mathbb{R}$.
\end{exer}
\begin{proof}
Rewriting $||f_n-f_k||_{C^1}<\epsilon$, we have
$$||f_n-f_k||_u+||f_n'-f_k'||_u<\epsilon$$
Because || denotes the supremum of an absolute value, both $||f_n-f_k||_u$ and $||f_n'-f_k'||_u$ are positive implying that they are also both less than $\epsilon$. By definition 6.1.12, this means that the $f_n$ is uniformly Cauchy and $f_n$ converges pointwise to $f$. Fixing $c\in[a,b]$ we see that $\{f_n(c)\}\rightarrowf(c)$. Thus, by theorem 6.2.10, $f_n$ converges uniformly to a continuously differentiable function $f$ on $[a,b]$.
\end{proof}

\end{document}