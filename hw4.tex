\documentclass[12pt,oneside]{amsart}
\linespread{2.4}
\usepackage[utf8]{inputenc}

\title{Math 414/514 Homework 4}
\author{Elena Yang}
\date{February 17, 2021}

% First, we'll show that $f\cdot g$ is bounded. By definition 3.2, let $f$ be bounded by $M_f$, or $|f|\leq M_f$ and let $g$ be bounded by $M_g$, or $|g|\leq M_g.$ Thus, $|f\cdot g|\leq M_fM_g$ which means $f\cdot g$ is also bounded.\\
%Now we'll show that $D_{f\cdot g}$ has measure zero. Suppose there exists a point $x\notin D_f\cup D_g$. Then, $x\notin D_f$ and $x\notin D_g$. This means that $f,g$ are both continuous at $x$, so $f\cdot g$ is also continuous at x by Lebl Proposition 3.2.5 (iii). Thus, $x\notin D_{f\cdot g}$. \\
%We know that $D_{f\cdot g}\subseteq D_f\cup D_g$. Since $D_{f\cdot g}$ is subset of a union of two measure zero sets, then it also has measure zero. Thus, by Riemann-Lebesgue $f\cdot g$ is integrable.

% Packages
\usepackage[T1]{fontenc}
\usepackage{amsmath,amsfonts,amssymb,amsthm}
\usepackage[letterpaper,margin=1.5in]{geometry}
\usepackage[pagebackref]{hyperref}
\usepackage{booktabs}
\usepackage{enumitem}
\usepackage{fancyhdr}
\usepackage{hyperref}
\usepackage{mathtools}
\usepackage{comment}
\pagestyle{fancy}

\newtheorem{theorem}[equation]{Theorem}
\newtheorem{claim}[equation]{Claim}
\newtheorem{lemma}[equation]{Theorem}
\newtheorem{corollary}[equation]{Theorem}
\newtheorem{conjecture}[equation]{Conjecture}
\newtheorem{question}[equation]{Question}
\theoremstyle{definition}
\newtheorem{definition}[equation]{Definition}
\theoremstyle{remark}
\newtheorem{exer}{Exercise}
\newtheorem{remark}[equation]{Remark}
\newtheorem{example}[equation]{Example}
\numberwithin{equation}{exer}
\newenvironment{answer}{\bigskip\noindent\emph{Answer.}}{\hfill$\diamond$\newline}

\begin{document}
\maketitle

\begin{exer}
Lebl 5.2.4: Prove the mean value theorem for integrals. That is, prove that if $f:[a,b]\rightarrow \mathbb{R}$ is continuous, then there exists a $c\in [a,b]$ such that $\int_{a}^{b}f=f(c)(b-a)$.
\end{exer}

\begin{proof}
Consider $F(x)=\int_{a}^{x}f(t)dt$.\\ By definition 1.1, $F'(x)=f(x)$ for every $x\in[a,b]$ and $F(b)-F(a)=\int_{a}^{b}f(t)dt$. \\
By definition 1.2 on $F$, we have $F(b)-F(a)=F'(c)(b-a)$ \\
Since $F'(x)=f(x)$, we have $F(b)-F(a)=f(c)(b-a)=\int_{a}^{b}f(t)dt$. \\
Since this $c$ exists in the open interval by def 1.1 and def 1.2, it also exists in the closed interval $[a,b]$.
\end{proof}
\newpage
\begin{definition}
The first part of the Fundamental Theorem of Calculus states: let $F:[a,b]\rightarrow\mathbb{R}$ be continuous and differentiable on $(a,b)$, Let $f\in\mathbb{R}[a,b]$  and $f(x)=F'(x)$ for $x\in(a,b)$, then $\int_{a}^{b}f=F(b)-F(a)$. 
\end{definition}
\begin{definition}
The Mean Value Theorem states: if $f:[a,b]\rightarrow\mathbb{R}$ is a continuous function differentiable on $(a,b)$, then there exists a $c\in(a,b)$ such that $f(b)-f(a)=f'(c)(b-a)$ 
\end{definition}

\newpage
\begin{exer}
TBB 6.7.4: Calculate $\omega_f(0)$ for each of the following functions: \\
a. \begin{equation*}
f(x) = \left\{
        \begin{array}{ll}
            x & \quad x \neq 0 \\
            4 & \quad x = 0
        \end{array}
    \right.
\end{equation*} \\
b. \begin{equation*}
f(x) = \left\{
        \begin{array}{ll}
            0 & \quad x \in \mathbb{Q} \\
            1 & \quad x \notin \mathbb{Q}
        \end{array}
    \right.
\end{equation*} \\
c. Note that $n$ is an integer. \begin{equation*}
f(x) = \left\{
        \begin{array}{ll}
            n & \quad x=\frac{1}{n} \\
            0 & \quad \text{otherwise}
        \end{array}
    \right.
\end{equation*}\\
d. \begin{equation*}
f(x) = \left\{
        \begin{array}{ll}
            \sin\frac{1}{x} & \quad x \neq 0 \\
            0 & \quad x=0
        \end{array}
    \right.
\end{equation*} \\
e. \begin{equation*}
f(x) = \left\{
        \begin{array}{ll}
            \sin\frac{1}{x} & \quad x \neq 0 \\
            7 & \quad x=0
        \end{array}
    \right.
\end{equation*} \\
f. \begin{equation*}
f(x) = \left\{
        \begin{array}{ll}
            \frac{1}{x}\sin\frac{1}{x} & \quad x \neq 0 \\
            0 & \quad x=0
        \end{array}
    \right.
\end{equation*} 
\end{exer}
\begin{answer}
\\
a. In any interval $(x-\delta,x+\delta)$, $\sup=\max(f(4,x+\delta))$ and $\inf=\min(f(x-\delta,4))=0$. When $x=0$, $\lim_{\delta\to 0}(x+\delta)=0$. Thus, we have $\max(0,4)-\min(0,4)=4-0=\boxed{4}$\\
b. In any interval $(x-\delta,x+\delta)$, $f$ has output values $[0]\cup[1]$. So $\sup f((x-\delta,x+\delta))=1$ and $\inf f((x-\delta,x+\delta))=0$. \\
$\omega_f((x-\delta,x+\delta))=1-(0)=1$ \\
$\omega_f(0)=\lim_{\delta\to0}\omega_f((x-\delta,x+\delta))=lim_{\delta\to0}1=\boxed{1}$ \\
c. In any interval $(x-\delta,x+\delta)$, $f$ has output values $[0,\infty)$. So $\sup f((x-\delta,x+\delta))=\infty$ and $\inf f((x-\delta,x+\delta))=0$. \\
$\omega_f((x-\delta,x+\delta))=\infty-(0)=\infty$ \\
$\omega_f(0)=\lim_{\delta\to0}\omega_f((x-\delta,x+\delta))=lim_{\delta\to0}\infty=\boxed{\infty}$  \\
d. In any interval $(x-\delta,x+\delta)$, $f$ has output values $[-1,1]\cup[7]$. So $\sup f((x-\delta,x+\delta))=1$ and $\inf f((x-\delta,x+\delta))=-1$. \\
$\omega_f((x-\delta,x+\delta))=1-(-1)=2$ \\
$\omega_f(0)=\lim_{\delta\to0}\omega_f((x-\delta,x+\delta))=lim_{\delta\to0}2=\boxed{2}$\\
e. In any interval $(x-\delta,x+\delta)$, $f$ has output values $[-1,1]\cup[7]$. So $\sup f((x-\delta,x+\delta))=7$ and $\inf f((x-\delta,x+\delta))=-1$. \\
$\omega_f((x-\delta,x+\delta))=7-(-1)=8$ \\
$\omega_f(0)=\lim_{\delta\to0}\omega_f((x-\delta,x+\delta))=lim_{\delta\to0}8=\boxed{8}$ \\
f. In any interval $(x-\delta,x+\delta)$, $f$ has output values $[0,\infty)$. So $\sup f((x-\delta,x+\delta))=\infty$ and $\inf f((x-\delta,x+\delta))=0$. \\
$\omega_f((x-\delta,x+\delta))=\infty-(0)=\infty$ \\
$\omega_f(0)=\lim_{\delta\to0}\omega_f((x-\delta,x+\delta))=lim_{\delta\to0}\infty=\boxed{\infty}$
\end{answer}
\newpage
\begin{exer}
TBB 8.6.2: Show that the product of two Riemann integrable functions is itself Riemann integrable.
\end{exer}
\begin{proof}
Let $f:[a,b]\rightarrow\mathbb{R}$ and $g:[a,b]\rightarrow\mathbb{R}$ be Riemann integrable. By definition 3.1, $f,g$ are both bounded and $D_f,D_g$ (set of discontinuites) have measure zero.\\
First, we'll show that $f\cdot g$ is bounded. By definition 3.2, let $f$ be bounded by $M_f$, or $|f|\leq M_f$ and let $g$ be bounded by $M_g$, or $|g|\leq M_g.$ Thus, $|f\cdot g|\leq M_fM_g$ which means $f\cdot g$ is also bounded.\\
Now we'll show that $D_{f\cdot g}$ has measure zero. Suppose there exists a point $x\notin D_f\cup D_g$. Then, $x\notin D_f$ and $x\notin D_g$. This means that $f,g$ are both continuous at $x$, so $f\cdot g$ is also continuous at x by Lebl Proposition 3.2.5 (iii). Thus, $x\notin D_{f\cdot g}$. \\
We want to show that $D_{f\cdot g}\subseteq D_f\cup D_g$. We will prove this using the contrapositive - if a point $k\notin D_f\cup D_g$ then $k\notin D_{f\cdot g}$. Since $k$ is in neither of the sets of discontinuities, then $f$ and $g$ are both continuous at $k$. By proposition 3.2.5 (iii), $f\cdot g$ is also continuous at $k$, which means that $k\notin D_{f\cdot g}$.  \\

Since $D_{f\cdot g}$ is subset of a union of two measure zero sets, it also has measure zero. Thus, by Riemann-Lebesgue, $f\cdot g$ is integrable.
\end{proof}
\begin{definition}
Riemann-Lebesgue Theorem: Let $f:[a,b]\rightarrow\mathbb{R}$. Then \\ $f\in\mathbb{R}[a,b]$ if and only if (1) $f$ is bounded, and (2) the set of points of discontinuity of $f$ has measure zero.
\end{definition}
\begin{definition}
Suppose $f:D\rightarrow\mathbb{R}$ is a function. We say $f$ is bounded if there exists a number $M$ such that $|f(x)|\leq M$ for all $x\in D$.
\end{definition}
\newpage
\begin{exer}
TBB 6.8.12: Show that the set of real numbers in the interval [0,1] that do not have a 7 in their infinite decimal expansion is of measure zero.
\end{exer}
\begin{proof}
We will construct a union of numbers that do not have a $7$ in their infinite decimal expansion by restricting first the tens, then the hundreds, and so on decimal place. \\
Our first set will be $S_1$ where we construct an interval of numbers without a $7$ in the tens place. This is: $$S_1=[0,0.7)\cup[0.8,1]$$
Here, we deleted $[0.7,0.8)$. Note that this is $0.9=\frac{9}{10^1}$ of the original interval of $(0,1)$. \\
We'll no construct a second set which is the union of these intervals but with the added constraint that the second digit cannot be $7$. This is: 
$$S_2=([0,0.07)\cup[0.08,0.1])\cup([0.1,0.17)]\cup[0.18,0.2])\cup([0.2,0.27)]\cup[0.28,0.3])$$
$$\cup([0.3,0.37)]\cup[0.38,0.4])\cup([0.4,0.47)]\cup[0.48,0.5])\cup([0.5,0.57)]\cup[0.58,0.6])$$ $$\cup([0.6,0.67)]\cup[0.68,0.7))\cup([0.8,0.87)]\cup[0.88,0.9])\cup([0.9,0.97)]\cup[0.98,0.1])$$
The total length of this interval is $9\cdot(\frac{9}{10^2})$. Note that the total length of set $S_n=(\frac{9}{10})^n$ because each time we're removing one digit ($7$) from a total of $10$ digits in each place value holder. This value approaches $0$ as $n\to\infty$, it has measure $0$.
\end{proof}

\end{document}